\documentclass[10pt,letterpaper,times]{article}%book, beamer
%Inicia preámbulo

%\include{tesis}


%\usepackage{fourier}
\usepackage{mathpazo}
\usepackage{antpolt}

% to change the fonts
\usepackage{cancel}
\usepackage{array}

\usepackage{pdfpages}
\usepackage[utf8]{inputenc}
%\usepackage[numbers]{natbib}
\usepackage[english]{babel}
\usepackage{mathptmx}
\usepackage{amsmath}
\usepackage{graphicx}

\usepackage{xcolor}

%\usepackage[margin=1.5cm]{geometry}
\usepackage{fancyhdr}
\usepackage{parskip}
\usepackage[colorlinks=false]{hyperref}
\usepackage{amsmath}
\usepackage{amsmath,amsfonts}
\usepackage{amssymb}
\usepackage{mathrsfs}
\usepackage{siunitx}
%\usepackage[usenames]{color}
\usepackage{multirow} % para las tablas
%\usepackage[spanish,es-tabla]{babel}
\usepackage{hieroglf}
\usepackage{caption}
\usepackage{subcaption}
\usepackage{wrapfig}
\usepackage{booktabs}
%\usepackage{slashbox}
%\usepackage{tcolorbox}
\usepackage{multicol}
\usepackage{fontenc}
\usepackage{tgbonum}

\usepackage{float}
\usepackage{booktabs}
\usepackage{indentfirst}


\usepackage{indentfirst}
\usepackage{tabularray}

\usepackage{lscape}

\usepackage{listings}
\usepackage{appendix}

\usepackage{multicol}
\setlength{\columnsep}{1cm}

\usepackage{nameref}

\usepackage{tikz}
\usetikzlibrary{datavisualization.formats.functions} % LaTeX and plain TeX

\usetikzlibrary[datavisualization]
\usepackage{pgfplots}
\pgfplotsset{compat=1.18}

\usepackage[siunitx, RPvoltages,american resistor]{circuitikz}


\usetikzlibrary {circuits.ee.IEC} 
\newcommand*{\TickSize}{2pt}%

\ctikzset{quadpoles/transformer core/height=3}
\usepgfplotslibrary{fillbetween}
\author{Mario Sepúlveda-Hernández }

\title{Inverter configuration}
\begin{document}
	\maketitle
	\begin{figure}[h!]
		\centering
		
		\begin{tikzpicture}
			\draw (0,0) node[op amp](opamp){};
			\draw ($(opamp.-)+(-4,0)$) node[left]{$V_{in}$} to [short,o-] ($(opamp.-)+(-3,0)$)
			to [american resistor=$R_{i}$] ($(opamp.-)+(-1,0)$)
			to [short,-*] ($(opamp.-)+(-0.5,0)$)
			node[below]{$v_{n}$} 
			to [short] (opamp.-);
			\draw (opamp.+) to [short] ($(opamp.+)+(-0.5,0)$)
			to [short] ($(opamp.+)+(-0.5,-1)$) node[tlground]{};
			\draw ($(opamp.-)+(-0.5,0)$) to [short] ($(opamp.-)+(-0.5,2)$)
			to [american resistor=$R_{f}$] ($(opamp.-)+(3.5,2)$) 
			to [short,-*] ($(opamp.out)+(1.12,0)$);
			\draw (opamp.out) to [short,-o] node[anchor=north west]{$V_{out}$}($(opamp.out)+(3,0)$);
		\end{tikzpicture}
		\caption{Operational Amplifier (inverter configuration)}
		\label{fig:opampInverter}
	\end{figure}

The Figure \ref{fig:opampInverter} represents the inverter configuration with an operational amplifier; the input signal $V_{in}$ will be inverted at the output $V_{out}$

\begin{align*}
	i_{i}&=i_{in}+i_{f} \\
	Z_{i}&\longrightarrow \infty \ \ \Longrightarrow i_{in}=0 \\
	i_{i}&=i_{f}\\
	i_{i}&=\dfrac{V_{in}-\cancelto{0}{v_{n}}}{R_{i}}=\dfrac{V_{in}}{R_{i}}\\
	i_{f}&=\dfrac{ \cancelto{0}{v_{n}}-V_{out} }{R_{f}}=-\dfrac{V_{out}}{R_{f}}\\
	\Longrightarrow \dfrac{V_{in}}{R_{i}}&=-\dfrac{V_{out}}{R_{f}}\\
	\Longrightarrow \dfrac{V_{out}}{V_{in}}&=-\dfrac{R_{f}}{R_{i}}=A_{v}
\end{align*}
\end{document}   